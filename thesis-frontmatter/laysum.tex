\chapter{Abstract}

Red Supergiant stars represent the most luminous stars in the infrared sky.
Their intrinsic luminosities combined with the low dust extinction observed
in this regime makes these objects very attractive to study in the near-IR.
Added to this is the fact that RSGs are necessarily young objects,
therefore these stars are tracers of recent star formation in extra-galactic
systems.
As the next generation of telescopes will be optimised for study in the near-IR,
it is clear that in the coming years RSGs will play a prominent role in the way
which astronomers probe the local universe and out to larger distances with
space-based observations.
Therefore, bettering our understanding of these objects now and developing the
tools which will allow us to take full advantage of the suite of instrumentation
which will become available in the near future is vital.
This thesis aims to further the understanding of RSGs by focusing on photometry
and spectroscopy at near-IR wavelengths.

% I develop a more robust criteria for selecting RSGs using near-IR photometry alone.
% This has the advantage of decreasing the dependence of optical colours which are
% known to select only a sub-sample of RSGs.
% This selection criteria is tested on data for $>$300 red stars in M33 and the
% efficiency of this technique is detailed.

I describe the implementation of the J-band analysis technique which is used to
derive stellar parameters using near-IR spectra of RSGs.
This implementation is tested on a variety of synthetic and real spectra and
is shown to work well across a broad range of input spectra.

Using this implementation of the analysis routines I esimate stellar parameters
for 14 RSGs in NGC\,2100, a young massive star cluster in the Large Magellanic Cloud.
I estimate the chemical and dynamical properties of this star cluster and find
that this cluster has must have expanded owing to stellar evolution on a slow
timescale.

Using the analysis routines described,
11 RSGs in NGC\,6822 are observed with the new K-band multi-object spectrograph
(KMOS) and have stellar parameters estimated.
The data reduction process with KMOS is described in detail, in particular
where the reduction has been optimised for our data.
The metallicity of these stars is compared to previous estimates and is shown
to be in good agreement.


KMOS spectra of RSGs in NGC\,55 is presented and stellar parameters
are derived for these stars.
These results are compared to previous estimates in this galaxy and the spatial
distribution of the chemical abundances is assessed.

Using the metallicity measurements made within these three galaxies I estiamte
the local mass-metallicity relationship using RSGs.
I compare this relationship to other estiamtes from within the local group and
comment on the calibration of the high-redshift mass-metallicity relationship.

% By developing a more robust selection criteria using near-IR colours we
% decrease our dependence on optical surveys where RSGs can be

% Using a new spectroscopic techinique to derive stellar parameters from Red Supergiants we derive stellar parameters for RSGs in NGC6822.
% With current state of the art KMOS instrument on the VLT, Chile, stellar paramaters from red Supergiants can be probed at distances of up to $\sim$5Mpc.
% With a similar instrument operating on the E-ELT, this technique is feasible out to the Virgo cluster.

% A new implementation of this technique is presented and tested on galaxies within the local group.

% Using KMOS stellar parameters have been derived in NGC6822 at 0.5Mpc which represents the lowest metallicitiy environment this technique has been used on to date.
% These results agree well with previous estimates of metallicitiy using the young stellar population of this galaxy.
% There appears to be no significant metallicitiy gradient within this galaxy and the idea that Red Supergiant temperatures do not vary with metallicity is introduced.

% At a greater distance, 20 RSGs in NGC\,55 (2.2Mpc) are studied.
% This new technique is implemented and tested on Red supergiants at further distances in the galaxy NGC55.
% This represents the largest distance at which individual Red Supergiant stars have been studied in this great detail.


% The selection of Red Supergiants using near-infrared photometry is investigated and developed by studying a large data set of red stars in the local grouo galaxy M33.
