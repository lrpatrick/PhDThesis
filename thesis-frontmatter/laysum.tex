
\chapter{Summary}
\singlespacing

Red Supergiant (RSG) stars are the brightest stars at infrared wavelengths.
Their intrinsic brightness combined with the fact that dust does not obscure observations at these wavelengths,
makes these objects very attractive to study in the infrared.
% Added to this, the next generation of telescopes will be optimised for study at these wavelengths; therefore,
% it is clear that, in the coming years, RSGs will play a prominent role in the way that astronomers probe the local Universe
% and out to larger distances with
% space-based observations.
To make best use of the future facilities, the tools to study RSGs must be developed today.
The aim of this thesis is to further the understanding of RSGs by focusing on measuring their chemical properties in different galaxies.

To do this, I develop an analysis technique that uses observations of RSGs to estimate their properties.
% I test this technique thoroughly and compare my results with those of similar techniques and show that this analysis is able to accurately and precisely measure the chemical properties of RSGs.
Using this analysis technique, I measure the chemistry and dispersal of a young cluster of stars in the Large Magellanic Cloud, one of the closest galaxies to the Milky Way. I find the chemical properties to be in good agreement for the chemical properties of RGSs with previous studies.
I also measure the mass the mass of the cluster, for the first time, using the dispersal of the cluster, and show that the mass is consistent with that found using other methods.

I then measure the chemical properties of 11 RSGs in a dwarf irregular galaxy with a turbulent history (NGC\,6822), which is 10 times further away than the Large Magellanic Cloud.
I find the chemical properties to be in good agreement with other young stars in this galaxy and I present weak evidence for variations in the distribution of the chemical elements in this galaxy, which requires further study.
% In addition, I show that the chemical properties of the young and old populations in this galaxy are well explained by a simple model, which is interesting, as this simple model is thought not to be applicable to this galaxy.

I then present observations of 22 RSGs in a large galaxy, four times more distant than NGC\,6822, which is located outside of our Local Group of galaxies.
I am able to rule out that the target RSGs are in binary systems and I estimate the properties of the targets using the analysis routine presented, where I find good agreement with previous studies of young stars in this galaxy.

I conclude this thesis by summarising the main results and present a first-look calibration of the relationship between two fundamental determinants of galaxy evolution: mass and chemical abundance.
In addition, using $\sim$80 RSGs, with surface temperatures estimated in a consistent way, I show that the temperature of RSGs does not appear to depend upon their chemical properties, in disagreement with current models.


% \bibliography{../journals,../books}
