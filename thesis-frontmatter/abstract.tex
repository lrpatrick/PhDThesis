\chapter{Abstract}

Red Supergiant (RSG) stars are the most luminous stars in the infrared sky.
Their intrinsic luminosities combined with the low dust extinction observed
in this regime makes these objects very attractive to study in the near-infrared (IR).
In addition, RSGs are necessarily young objects,
as, they are tracers of recent star formation in extra-galactic
systems.
As the next generation of telescopes will be optimised for study in the near-IR,
it is clear that, in the coming years, RSGs will play a prominent role in the way that astronomers probe the local Universe and out to larger distances with
space-based observations.
Therefore, it is vital to better our understanding of these objects now and develop the
tools that will allow us to take full advantage of the suite of instrumentation
that will become available in the near future.
This thesis aims to further the understanding of RSGs by focusing on quantitative studies of near-IR spectroscopic observations.

To this end, I develop an analysis technique that uses spectroscopic and photometric observations to estimate stellar parameters of RSGs.
The observations are compared with synthetic spectra extracted from stellar model atmospheres, where departures from local thermodynamic equilibrium have been calculated for the diagnostic spectral lines.
This technique is tested thoroughly on synthetic and real observations and is shown to reliably estimate stellar parameters in both regimes when compared with input parameters and previous studies respectively.


Using the analysis routines developed in Chapter~\ref{ch:janal}, in Chapter~\ref{ch:ngc2100} I measure the chemistry and kinematics of NGC\,2100, a young massive cluster (YMC) of stars in the Large Magellanic Cloud, using near-IR spectroscopic observations of 14 RSGs taken with the new $K$-band multi-object spectrograph (KMOS).
I estimate the average metallicity to be $-$0.43\,$\pm$\,0.10\,dex, which is in good agreement with previous studies.
I compare the observed location of the target RSGs on the Hertzsprung--Russell diagram with that of a Solar-like metallicity YMC and show that there appears to be no significant difference in the appearance of the RSGs in these two clusters.
By combining the individual RSG spectra, I create an integrated-light cluster spectrum and show that the stellar parameters estimated, using the same technique as for individual RSGs, are in good agreement with the average properties of the cluster.
In addition, I measure -- for the first time -- an upper limit of the dynamical mass of NGC\,2100 to be 15.2~$\times$~10$^4$\,M$_{\odot}$, which is consistent with the literature measurement of the photometric mass of the cluster.


In Chapter~\ref{ch:ngc6822}, I present observations of RSGs in NGC\,6822, a dwarf irregular with a turbulent history, observed with KMOS.
The data reduction process with KMOS is described in detail, in particular
where the reduction has been optimised for the data.
Stellar parameters are estimated using the technique presented in Chapter~\ref{ch:janal} and an average metallicity in NGC\,6822 of $-$0.55\,$\pm$\,0.13\,dex is found, consistent with previous measurements of young stars in this galaxy.
The spatial distribution of metallicity is estimated and weak evidence is found for a radial metallicity gradient, which will require follow-up observations.
In addition, I show that the metallicities of the young and old populations of NGC\,6822 are well explained using a simple closed-box chemical evolution model, an interesting result, as NGC\,6822 is expected to have undergone significant recent interactions.


In Chapter~\ref{ch:ngc55}, I present multi-epoch KMOS observations of 22 RSGs in the Sculptor Group galaxy NGC\,55.
Radial velocities are measured for the sample and are shown to be in good agreement with previous studies. Using the multi-epoch data, I find no evidence for radial velocity variables within the sample.
Stellar parameters are estimated for 10 targets and are shown to be in good agreement with previous estimates.

I conclude this thesis by summarising the main results and present a first-look calibration of the relationship between galaxy mass and metallicity using RSGs.
By comparing the RSG metallicity estimates to metallicities estimated from $\sim$50\,000 Sloan digital sky survey galaxies, I show that the absolute metallicities of the two samples disagree. A more quantitative analysis requires additional RSG observations.

In addition, using $\sim$80 RSGs, with stellar parameters estimated in a consistent way, I show that there appears to be no dependence of the temperature of RSGs upon metallicity.
This is in disagreement with current evolutionary models, which display a temperature change of $\sim$450\,K over the studied range in metallicity.

Finally, I outline potential areas for future work, focusing on follow-up studies that have been identified as a result of the work done in this thesis.

% \bibliography{../journals,../books}
