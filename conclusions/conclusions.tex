\chapter{Conclusions}
\label{ch:conclusions}

\section{Summary} % (fold)
\label{sec:exec-sum}
In this thesis I have used near-IR spectroscopic observations to measure the chemical abundance of RSGs in different environments within the Local Universe.
I have detailed the background theory surrounding the evolution of massive stars and their pivotal role in shaping the Universe.
In order to make the best use of the extremely luminous nature of RSGs, I have described and detailed KMOS: the only near-IR multi-object spectrograph in the suthern-hemisphere.

I have developed and tested a grid-based analysis technique to estiamte stellar parameters of RSGs using medium-resolution $J$-band spectroscopy.
I estimate parameters by sampling the posterior probability density function using a maximum-likelihood approach and shown that this technique is not only internally consistent but is also in good agreement with literature measurements.

In Chapter~\ref{ch:ngc2100} this technique is applied to measure the chemistry and kinematics of a YMC in the LMC: NGC\,2100.
KMOS spectra of 14 RSGs in NGC\,2100 are used to estimate the dynamical properties of this star cluster for the first time.
This study demonstrates that KMOS can be used to measure velocities of stars in external galaxies to a precision of $<$5\,\kms and places an upper limit on the line-of-sight velocity dispersion of NGC\,2100 at $\sigma_{1D}$~=~3.9\,\kms, at the 95\% confidence level, where no evidence is found for spatial variations in this estimate.
Using this upper limit, an upper limit to the dynamical mass of NGC\,2100 has been calculated (assuming virial equilibrium) as $M_{dyn}$~=~$15.2\times 10^{4}M_{\odot}$, in good agreement with the literature value of the photometric mass for NGC\,2100.

The chemistry of NGC\,2100 has been estimated using the analysis technique developed in this thesis where the average present-day metallicity of NGC\,2100 is [Z]~=~$-$0.43\,$\pm$\,0.10\,dex, which agrees well with previous studies in this cluster and with studies of the young stellar population of the LMC.

The observational properties of the RSGs in NGC\,2100 are compared with a star cluster of similar age and mass at Solar-like metallicity.
This comparison allows differences in the observational properties of RSGs at different metallicities to be assessed and I show that there appears to be no significant difference bewtween these Solar-like and LMC-like metallicity clusters.

As the RSGs population dominates the infrared light output of a YMC, of a particular age and mass, I combine the RSG spectra to create a simulated integrated-light cluster-spectum of NGC\,2100.
Using the same analysis technique I demonstrate that the stellar parameters estiamted using integrated-light spectroscopy of YMCs are representivive of the average parameters of the RSG population within the cluster.

In Chapter~\ref{ch:ngc6822} KMOS spectroscopy of 18 RSGs in the dwarf irregular galaxy NGC\,6822 is presented.
The KMOS data used in this chapter was obtained using KMOS-SV time before this instrument was released to the general community.
This chapter represents the first KMOS study of RSGs and forms the base of which all other studies in this area are built from.
In order to characterise the performance of the data reduction I have reduced and analysed the data using two different methods of telluric correction: the more time expensive 24-arm telluric correction and the more efficient three-arm telluric correction method.
Both methods give consistent results and the 3-arm telluric correction is shown to work as effectively (in most cases).
However, we caution, in the low signal-to-noise regime, the 24-arm telluric method will give more reliable results.

Stellar parameters are calculated using two different analysis techniques and are shown to agree well.
The present day metallicity of NGC\,6822 from RSGs is
[Z]~=~$-0.55\pm0.13\,dex$ which is consistent with previous measurements of the young stellar population in NGC\,6822.
The data show evidence for a low-significance abunance gradient within NGC\,6822: the first of its kind in a dwarf irregular galaxy.
A larger follow-up study is required in order to fully characterise this gradient.

The chemical abundancnes of the young and old stellar populations are well explained by a simple closed-box chemical evolution model.
However, while an interesting result, we note that the closed-box model is unlikely to be a good assumption for this galaxy given its morphology.

The effective temperatures of RSGs are compared in four galaxies using the same analysis technique.
These environments span 0.55\,$dex$ in metallicity (Solar to SMC) and no evidence is found for a significant variation in temperature with respect to metallicity.
This is in contrast with evolutionary models which, for a  similar change in metallicity, produces a shift in the temperature of RSGs of up to 450\,K.
In addition, in this chapter I argue that the observed shift the spectral type of RSGs (defined at optical wavelengths) with respect to metallicity does not imply that the temperatures of RSGs is dependent upon metallicity.



% section summary (end)

\section{Future Projects} % (fold)
\label{sec:future_projects}

% section future_projects (end)

\section{Closing Remarks} % (fold)
\label{sec:closing_remarks}

% section closing_remarks (end)